\documentclass[letterpaper, 12pt]{article}

\usepackage[margin=1in]{geometry}
\setlength{\parindent}{0pt}% No paragraph indent
\pagestyle{empty}% No page headers/footers
\begin{document}

\hfill
\begin{tabular}{r@{}}
  \today\\[12pt]
  Taro Naoi\\
  Department of Materials Science and Engineering\\
  Cornell University\\
  Ithaca, New York 14853, USA\\
  Email: taronaoi@gmail.com
\end{tabular}

\bigskip% or \vspace{20pt}

\begin{tabular}{@{}l}
Physical Review B
\end{tabular}

\bigskip

Dear Editors,

\bigskip

We wish to submit our manuscript entitled ``Atomic structure determination of mixed amorphous Ta$_{2}$O$_{5}$-GeO$_{2}$ thin films deposited on silicon nitride membranes'' for consideration by \emph{Physical Review B}. We outline the significance of our findings below and explain why this manuscript deserves serious consideration for publication in Physical Review B.

\bigskip

Ta$_{2}$O$_{5}$ has found a wide variety of uses owing to its high dielectric constant, for instance as capacitors in DRAM applications and more recently, in non-volatile memories. In our previous study, we found that introducing a small amount of GeO$_{2}$ into amorphous Ta$_{2}$O$_{5}$ enhanced the dielectric constant of the material at low GeO$_{2}$ concentrations. We deteremined that dielectric enhancement in Ta$_{2}$O$_{5}$-rich compositions was a direct consequence of enhanced ionic polarizabilities. However, an explicit structure-property relationship was not established.

\bigskip

In the present work, we build on our previous study by probing the local structure of amorphous Ta$_{2}$O$_{5}$-GeO$_{2}$ thin films over a range of compositions, to establish a clear structure-property relationship. We apply various experimental tools to reduce the structure space that must be explored because we cannot simply apply refinement techniques that rely on crystal symmetry. Furthermore, we codeposit films on silicon nitride membranes, which greatly facilitates background subtraction of peaks caused by Si thermal diffuse scattering, and allows us to probe multiple sample compositions generated in one single experiment. Our study shows that the dielctric anomalies in Ta$_{2}$O$_{5}$-GeO$_{2}$ are in fact a consequence of local structural changes.

\bigskip

We believe that our results provide fundamental insight into how local atomic structure leads to changes in dielectric properties. Understanding the fundamental structure-property relationship of complex amorphous thin films should prove useful to a wide readership, thus we feel that our paper is suitable for publication in Physical Review B.

\bigskip

Sincerely,

\bigskip

Taro Naoi

\end{document}